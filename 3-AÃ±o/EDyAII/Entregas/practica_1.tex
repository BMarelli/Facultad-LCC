\documentclass[12]{article}
\usepackage{amsmath}
\usepackage{amssymb}
\begin{document}
  \title{Resolución Practica 1}
  \author{Bautista Marelli}
  \date{Abril 2, 2020}
  \maketitle
  \section{Ejercicio 7 - b}
  Tenemos como recurrencia: $T(n)\ =\ T(n - 1)\ +\ n$

  Para resolver este ejercicio usamos un árbol de recurrencia. En este caso, el árbol
  es lineal.

  Obtenemos que, $T(n)\ = \sum_{i = 0}^{n - 2} (n-i)\ + T(1)$
      $$T(n)\ = \sum_{i = 0}^{n - 2} (n-i)\ - 1 + T(1)$$
      $$T(n)\ = \sum_{i = 2}^{n} (i)\ - 1+ T(1)$$
      $$T(n)\ = \frac{n*(n+1)}{2}\ - 1 + T(1)$$
      $$T(n)\ = \frac{n^2 + n}{2} \  - 1 + T(1)$$
  
  Y como podemos ver, $T(n)\ = \frac{n^2 + n}{2} \  - 1 + T(1) \ \in O(n^2)$
  
\end{document}
