\documentclass[titlepage, 12pt]{article}
\usepackage[a4paper, margin=1in]{geometry}
\usepackage[utf8]{inputenc}
\usepackage[spanish,mexico]{babel}
\usepackage{graphicx}
\usepackage{hyperref}
\usepackage{tabularx}
\usepackage{float}

% \usepackage{indentfirst}

\renewcommand{\baselinestretch}{1.25}
\renewcommand{\contentsname}{Índice}

\def\sectionautorefname{Sección}

\begin{document}
\title{Trabajo Práctico 1}
\author{Juan Cruz de La Torre, Bautista Marelli}
\date{17 de abril, 2020}

\maketitle

\tableofcontents
\newpage

\section{Introducción}
\label{sec:introduccion}

El objetivo de este informe es presentar un análisis sobre los datos registrados en el Censo Forestal Urbano Público del año 2011, realizado en dos comunas del sur de Buenos Aires con el objetivo de contabilizar y determinar el estado actual del arbolado urbano público.

En dicho censo, se registraron las variables descritas a continuación.

\begin{table}[H]
  \begin{tabularx}{\linewidth}{|l|X|}
    \hline
    \textbf{Nombre} & \textbf{Descripción}                                                                                                      \\ \hline
    ID              & Identificación del árbol.                                                                                                 \\ \hline
    altura          & Altura de cada árbol, medida en metros (m).                                                                               \\ \hline
    diámetro        & Diámetro de cada árbol, medida en centímetros (cm).                                                                       \\ \hline
    inclinación     & Ángulo que forma el tronco del árbol respecto a una perpendicular al suelo, medido en grados ($^{\circ}$).                \\ \hline
    especie         & Especie a la que pertenece el árbol: Eucalipto, Jacarandá, Palo borracho, Casuarina, Fresno, Ceibo, Ficus, Álamo, Acacia. \\ \hline
    origen          & Procedencia de la especie: Exótico, Nativo/Autóctono, No Determinado.                                                     \\ \hline
    brotes          & Número de brotes jóvenes crecidos durante el último año.                                                                  \\ \hline
  \end{tabularx}

  \caption{Variables registradas en el Censo Forestal Urbano Público 2011.}
  \label{tbl:variables}
\end{table}

\newpage
\section{Análisis descriptivo}

TODO: En esta sección se hace x, y, z.

\subsection{Análisis univariado}

TODO: por cada una de las variables cualitativas, agregar piechart y texto

TODO: por cada una de las variables cuantitativas, agregar histograma/barras, poligono acumulativo y texto

TODO: agregar a los cuadros titulos, labels de axis, mas numeritos en los ejes, y labels+captions a las tablas / graficos.

\subsubsection{Origen}

\begin{table}[H]
  \begin{tabularx}{\textwidth}{|l|X|X|}
    \hline
    Origen           & frecAbs & frecRel \\ \hline
    Exótico          & 250     & 0.71    \\ \hline
    Nativo/Autóctono & 100     & 0.29    \\ \hline
  \end{tabularx}
\end{table}

\subsubsection{Especie}

\begin{table}[H]
  \begin{tabularx}{\textwidth}{|l|X|X|}
    \hline
    Especie       & frecAbs & frecRel \\ \hline
    Acacia        & 12      & 0.03    \\ \hline
    Álamo         & 66      & 0.19    \\ \hline
    Casuarina     & 46      & 0.13    \\ \hline
    Ceibo         & 12      & 0.03    \\ \hline
    Eucalipto     & 73      & 0.21    \\ \hline
    Ficus         & 17      & 0.05    \\ \hline
    Fresno        & 38      & 0.11    \\ \hline
    Jacarandá     & 41      & 0.12    \\ \hline
    Palo borracho & 45      & 0.13    \\ \hline
  \end{tabularx}
\end{table}

\subsubsection{Altura}

\begin{table}[H]
  \begin{tabularx}{\textwidth}{|c|X|X|X|X|}
    \hline
            & frecAbs & frecRel & frecAbsAcum & frecRelAcum \\ \hline
    [0,3)   & 8       & 0.02    & 8           & 0.02        \\ \hline
    [3,6)   & 20      & 0.06    & 28          & 0.08        \\ \hline
    [6,9)   & 40      & 0.11    & 68          & 0.19        \\ \hline
    [9,12)  & 45      & 0.13    & 113         & 0.32        \\ \hline
    [12,15) & 49      & 0.14    & 162         & 0.46        \\ \hline
    [15,18) & 70      & 0.20    & 232         & 0.66        \\ \hline
    [18,21) & 49      & 0.14    & 281         & 0.80        \\ \hline
    [21,24) & 39      & 0.11    & 320         & 0.91        \\ \hline
    [24,27) & 13      & 0.04    & 333         & 0.95        \\ \hline
    [27,30) & 6       & 0.02    & 339         & 0.97        \\ \hline
    [30,33) & 7       & 0.02    & 346         & 0.99        \\ \hline
    [33,36) & 3       & 0.01    & 349         & 1.00        \\ \hline
    [36,39) & 1       & 0.00    & 350         & 1.00        \\ \hline
  \end{tabularx}
\end{table}

\subsubsection{Diámetro}

\begin{table}[H]
  \begin{tabularx}{\textwidth}{|c|X|X|X|X|}
    \hline
            & frecAbs & frecRel & frecAbsAcum & frecRelAcum \\ \hline
    [0,3)   & 257     & 0.73    & 257         & 0.73        \\ \hline
    [3,6)   & 23      & 0.07    & 280         & 0.80        \\ \hline
    [6,9)   & 16      & 0.05    & 296         & 0.85        \\ \hline
    [9,12)  & 23      & 0.07    & 319         & 0.91        \\ \hline
    [12,15) & 6       & 0.02    & 325         & 0.93        \\ \hline
    [15,18) & 9       & 0.03    & 334         & 0.95        \\ \hline
    [18,21) & 8       & 0.02    & 342         & 0.98        \\ \hline
    [21,30) & 5       & 0.01    & 347         & 0.99        \\ \hline
    [30,60) & 3       & 0.01    & 350         & 1.00        \\ \hline
  \end{tabularx}
\end{table}

\subsubsection{Inclinación}

\begin{table}[H]
  \begin{tabularx}{\textwidth}{|c|X|X|X|X|}
    \hline
             & frecAbs & frecRel & frecAbsAcum & frecRelAcum \\ \hline
    [0,3)    & 257     & 0.73    & 257         & 0.73        \\ \hline
    [3,6)    & 23      & 0.07    & 280         & 0.80        \\ \hline
    [6,9)    & 16      & 0.05    & 296         & 0.85        \\ \hline
    [9,12)   & 23      & 0.07    & 319         & 0.91        \\ \hline
    [12,15)  & 6       & 0.02    & 325         & 0.93        \\ \hline
    [15,18)  & 9       & 0.03    & 334         & 0.95        \\ \hline
    [18,21)  & 8       & 0.02    & 342         & 0.98        \\ \hline
    [21,30)  & 5       & 0.01    & 347         & 0.99        \\ \hline
    [30, 45) & 0       & 0.00    & 347         & 0.99        \\ \hline
    [30,60)  & 3       & 0.01    & 350         & 1.00        \\ \hline
  \end{tabularx}
\end{table}

\subsubsection{Brotes}

\begin{table}[H]
  \begin{tabularx}{\textwidth}{|c|X|X|X|X|}
    \hline
      & frecAbs & frecRel & frecAbsAcum & frecRelAcum \\ \hline
    0 & 1       & 0.00    & 1           & 0.00        \\ \hline
    1 & 15      & 0.04    & 16          & 0.05        \\ \hline
    2 & 59      & 0.17    & 75          & 0.21        \\ \hline
    3 & 116     & 0.33    & 191         & 0.55        \\ \hline
    4 & 74      & 0.21    & 265         & 0.76        \\ \hline
    5 & 42      & 0.12    & 307         & 0.88        \\ \hline
    6 & 30      & 0.09    & 337         & 0.96        \\ \hline
    7 & 11      & 0.03    & 348         & 0.99        \\ \hline
    8 & 2       & 0.01    & 350         & 1.00        \\ \hline
  \end{tabularx}
\end{table}

\subsection{Análisis bivariado}

\subsubsection{Origen + Especie}

\subsubsection{Brotes + Origen}

\subsubsection{Altura + Especie}

\subsubsection{Diámetro + Especie}

\subsubsection{Inclinación + Especie}

\subsubsection{Brotes + Especie}


\end{document}
