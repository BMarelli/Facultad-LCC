\documentclass{article}
\usepackage[utf8]{inputenc}
\usepackage{graphicx}
\usepackage{util/czt}

\begin{document}
\begin{figure}
\centering
\includegraphics[width=5cm]{util/Logo-UNR.png}
\end{figure}

\title{Ingeniería de Software - Practica 1}
\author{Bautista Marelli}
\date{\today}

\maketitle
\newpage

\section{Apartado 1}
  \begin{zed}
    ESTADO ::= on | off
  \end{zed}
  \begin{schema}{Sistema}
    lamp: ESTADO
  \end{schema}
  \begin{schema}{SistemaInicial}
    lamp: ESTADO
    \where
    lamp = off
  \end{schema}
  \begin{schema}{PrenderB1}
    \Delta Sistema \\
    lamp: ESTADO \\
    \where
    lamp == off \\
    lamp' = on
  \end{schema}
  \begin{schema}{ApagarB1}
    \Delta Sistema \\
    lamp: ESTADO
    \where
    lamp == on \\
    lamp' = off
  \end{schema}
  Idem con el boton 2
\section{Apartado 2}
\begin{zed}
  [MENSAJES] \\
  ESTADO ::= enviando | recibiendo \\
  BUFFER ::= \langle MENSAJES \rangle \\
  ERROR ::= si | no \\
  PARIEDADOK ::= P(MENSAJES)
\end{zed}
\begin{schema}{Dispacher}
  buff: BUFFER \\
  estado: ESTADO
\end{schema}
\begin{schema}{DispacherInicial}
  buff: \langle \rangle \\
  estado: recibiendo
\end{schema}
\begin{schema}{Canal1Ok}
  \Delta Dispacher \\
  m1?: MENSAJES \\
  rep1!: ERROR
  \where
  \# buff \leq 9 \\
  m? \in PARIEDADOK \\
  estado: recibiendo \\
  buff' = buff \dcat m1? \\
  estado' = estado \\
  resp1: no
\end{schema}
\begin{schema}{Canal1Error}
  \Xi Dispacher \\
  m1?: MENSAJES \\
  buff: BUFFER \\
  estado: ESTADO \\
  rep1: ERROR
  \where
  \# buff == 10 \cup m1? \notin PARIEDADOK \\
  buff' = buff \\
  estado' = estado \\
  rep1: si
\end{schema}
\begin{schema}{VaciarBuffer}
  \Delta Dispacher \\
  rep1?: MENSAJES \\
  \where
  \# buff == 10 \\
  estado = enviando \\
  rep1 = head\ buff \\
  estado' = recibiendo \\
  buff' = tail\ buff
\end{schema}
\begin{zed}
  Canal1 = Canal1Ok \cup Canal1Error \cup VaciarBuffer
\end{zed}

\section{Apartado 3}
Especificamos nuestros tipos:
\begin{zed}
  [PERSONAS] \\
  DIRECTOR: P(PERSONAS) \\
  GUIONISTA: P(PERSONAS)
\end{zed}

\end{document}
