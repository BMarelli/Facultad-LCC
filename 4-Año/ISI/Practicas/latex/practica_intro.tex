\documentclass{article}
\usepackage[utf8]{inputenc}
\usepackage{graphicx}
\usepackage{util/czt}

\begin{document}
\begin{figure}
\centering
\includegraphics[width=5cm]{util/Logo-UNR.png}
\end{figure}

\title{Ingeniería de Software - Practica Apunte Z}
\author{Bautista Marelli
}
\date{\today}

\maketitle
\newpage

\begin{enumerate}
  \item 
    \begin{zed}
      MESAJES ::= ok | clienteInexistente | montoNulo \\
                | noPoseeSaldoSuficiente | saldoNoNulo 
    \end{zed}
  \item ?
  \item ?
  \item 
    \begin{schema}{NuevoClienteOk}
      \Delta Banco \\
      d? : DNI \\
      rep! : MESAJES
      \where
      d? \notin \dom ca \\
      ca' = ca \cup \{d? \mapsto 0\} \\
      rep! = ok
    \end{schema}

    \begin{schema}{ClienteExiste}
      \Xi Banco \\
      d? : DNI \\
      rep! : MESAJES
      \where
      d? \in \dom ca \\
      rep! = numeroClienteEnUso
    \end{schema}

    \begin{schema}{MontoRequeridoOk}
      \Delta Banco \\
      d? : DNI \\
      m? : DINERO \\
      rep! : MENSAJE
      \where
      d? \in \dom ca \\
      m? >= 200 \\
      ca' = ca \oplus \{ d? \mapsto (ca \; d?) + m? \} \\
      rep! = ok
    \end{schema}
    
    \begin{schema}{MontoRequeridoError}
      \Xi Banco \\
      m? : DINERO \\
      rep! : MENSAJE
      \where
      m? < 200 \\
      rep! = montoInsuficiente
    \end{schema}

    \begin{zed}
      MontoRequerido == MontoRequeridoOk \cup MontoRequeridoError
    \end{zed}

    \begin{zed}
      NuevoCliente = (NuevoClienteOk \lor ClienteExiste) \land MontoRequerido 
    \end{zed}
    \item 
      \begin{zed}
        MENSAJE ::= ok | clienteInexistente | montoNulo \\
                  | noPoseeSaldoSuficiente | saldoNoNulo | numeroClienteEnUso \\
                  | montoInsuficiente
      \end{zed}
    \item
      
\end{enumerate}


\end{document}
