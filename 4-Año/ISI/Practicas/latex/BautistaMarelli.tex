\documentclass[a4paper, 12pt] {article}
\usepackage{czt}
\usepackage[margin=0.5in]{geometry}

\title{Parcial 1}
\author{Bautista Marelli}
\date{\today}

\begin{document}

\maketitle

El chequeo de tipos se realizo utilizando el plugin Community Z Tools en Eclipse

\begin{itemize}
\item m es un mail de cliente $\approx$ MAIL(m)
\item p es un producto $\approx$ NOMBREPRODUCTO(p)
\item c es un carro de compra $\approx$ Carro(c)
\item cv es un carro de compra vacio $\approx$ CarroVacio(cv)
\item s es un sistema de compra $\approx$ SistemaDeCompra(s)
\item res es un mensaje resportado por el sistema indicando si una operacion se realizo exitosamente o hubo un error $\approx$ RES(res)
\item Mensaje de exito de una operacion $\approx$ ok
\item Mensaje de error de un cliente no iniciado al sistema $\approx$ errorClienteNoIniciado
\item Mensaje de error de un cliente ya iniciado al sistema $\approx$ errorClienteIniciado
\item Mensaje de error de faltante de un producto en stock $\approx$ errorNoHayStock
\item Mensaje de error de un producto que no se encuentra en el carro del cliente $\approx$ errorProductoNoExisteEnCarro
\item Mensaje de error de un producto que ya se encuentra en el carro del cliente $\approx$ errorProductoExiteEnCarro
\item cant es la cantidad de un producto $\approx$ CANTIDAD(cant)
\item Operacion para agregar una cantidad de un producto a un carro $\approx$ CAgregarProducto
\item Operacion para eliminar un producto a un carro $\approx$ CEliminarProducto
\item Operacion para editar una cantidad de un producto a un carro $\approx$ CEditarCantidadProducto
\item Operacion para un usuario para agregar una cantidad de un producto a su carro $\approx$ SCAgregarProducto
\item Operacion para un usuario para eliminar un producto a su carro $\approx$ SCEliminarProducto
\item Operacion para un usuario para editar una cantidad de un producto a su carro $\approx$ SCEditarProducto
\item Operacion para un usuario para iniciar al sistema $\approx$ SCIniciarCompra
\item Operacion para un usuario para finalizar su compra en el sistema $\approx$ SCFinalizarCompra
\end{itemize}

\begin{zed}
[ MAIL, NOMBREPRODUCTO ] \\
RES ::= ok | errorClienteNoIniciado | errorClienteIniciado \\
\t1 | errorNoHayStock | errorProductoNoExisteEnCarro \\
\t1 | errorProductoExiteEnCarro \\
CANTIDAD == \nat
\end{zed}

\begin{schema}{Carro}
productos: NOMBREPRODUCTO \pfun CANTIDAD
\end{schema}

\begin{schema}{SistemaDeCompra}
clientes: MAIL \pfun Carro \\
stock: NOMBREPRODUCTO \pfun CANTIDAD
\end{schema}

\begin{schema}{CAgregarProductoOk}
\Delta Carro \\
nombProd?: NOMBREPRODUCTO \\
cant?: CANTIDAD
\where
nombProd? \notin \dom productos \\
cant? > 0 \\
productos' = productos \cup \{ nombProd? \mapsto cant? \}
\end{schema}

\begin{schema}{CAgregarProductoErrorCantidadNula}
\Xi Carro \\
nombProd?: NOMBREPRODUCTO \\
cant?: \nat
\where
nombProd? \notin \dom productos \\
cant? = 0
\end{schema}

\begin{zed}
CAgregarProducto == CAgregarProductoOk \\
\t1 \lor CAgregarProductoErrorCantidadNula
\end{zed}

\begin{schema}{CEliminarProducto}
\Delta Carro \\
nombProd?: NOMBREPRODUCTO
\where
productos' = \{ nombProd? \} \ndres productos
\end{schema}

\begin{schema}{CEditarCantidadProducto}
\Delta Carro \\
nombProd?: NOMBREPRODUCTO \\
cant?: CANTIDAD
\where
nombProd? \in \dom productos \\
cant? > 0 \\
productos' = productos \oplus \{ nombProd? \mapsto cant? \}
\end{schema}

\begin{schema}{ErrorClienteNoIniciado}
\Xi SistemaDeCompra \\
mail?: MAIL \\
res!: RES
\where
mail? \notin \dom clientes \\
res! = errorClienteNoIniciado
\end{schema}

\begin{schema}{CarroASistemaDeCompra}
\Delta SistemaDeCompra \\
\Delta Carro \\
mail?: MAIL
\where
mail? \in \dom clientes \\
clientes \: mail? = \theta Carro \\
clientes' = clientes \oplus \{ mail? \mapsto \theta Carro\ ' \} \\
\end{schema}

\begin{schema}{SCAgregarProductoOk}
CarroASistemaDeCompra \\
CAgregarProducto \\
res!: RES
\where
nombProd? \in \dom stock \land (stock \: nombProd?) \geq cant? \\
nombProd? \notin \dom (clientes \: mail?).productos \\
stock' = stock \oplus \{ nombProd? \mapsto (stock \: nombProd?) - cant? \} \\
res! = ok
\end{schema}

\begin{schema}{SCAgregarProductoErrorNoHayStock}
\Xi SistemaDeCompra \\
CarroASistemaDeCompra \\
nombProd?: NOMBREPRODUCTO \\
cant?: CANTIDAD \\
res!: RES
\where
nombProd? \notin \dom stock \lor (stock \: nombProd?) < cant? \\
res! = errorNoHayStock
\end{schema}

\begin{schema}{SCAgregarProductoErrorExisteProducto}
\Xi SistemaDeCompra \\
CarroASistemaDeCompra \\
nombProd?: NOMBREPRODUCTO \\
cant?: CANTIDAD \\
res!: RES
\where
nombProd? \in \dom stock \land (stock \: nombProd?) \geq cant? \\
nombProd? \in \dom (clientes \: mail?).productos \\
res! = errorProductoExiteEnCarro
\end{schema}

\begin{zed}
SCAgregarProductoError == SCAgregarProductoErrorNoHayStock \\
\t1 \lor SCAgregarProductoErrorExisteProducto \\
SCAgregarProducto == ErrorClienteNoIniciado \\
\t1 \lor SCAgregarProductoOk \\
\t1 \lor SCAgregarProductoError  \\
\end{zed}

\begin{schema}{ErrorProductoNoExisteEnCarro}
\Xi SistemaDeCompra \\
CarroASistemaDeCompra \\
nombProd?: NOMBREPRODUCTO \\
res!: RES
\where
nombProd? \notin \dom (clientes \: mail?).productos \\
res! = errorProductoNoExisteEnCarro
\end{schema}

\begin{schema}{SCEliminarProductoOk}
CarroASistemaDeCompra \\
CEliminarProducto \\
res!: RES
\where
nombProd? \in \dom (clientes \: mail?).productos \\
stock' = \\
\t1 stock \\
\t1 \oplus \{ nombProd? \mapsto (stock \: nombProd?) + (clientes \: mail?).productos \: nombProd? \} \\
res! = ok
\end{schema}

\begin{zed}
SCEliminarProducto == ErrorClienteNoIniciado \\
\t1 \lor SCEliminarProductoOk \\
\t1 \lor ErrorProductoNoExisteEnCarro
\end{zed}

\begin{schema}{SCEditarDisminuirProductoOk}
CarroASistemaDeCompra \\
CEditarCantidadProducto \\
res!: RES
\where
nombProd? \in \dom (clientes \: mail?).productos \\
nombProd? \in \dom stock \\
cant? > 0 \\
cant? \leq ((clientes \: mail?).productos \: nombProd?) \\
stock' = \\ 
\t1 stock \\
\t1 \oplus \{ nombProd? \mapsto \\
\t2 (stock \: nombProd?) + (((clientes \: mail?).productos \: nombProd?) - cant?) \} \\
res! = ok
\end{schema}

\begin{schema}{SCEditarAumentarProductoOk}
CarroASistemaDeCompra \\
CEditarCantidadProducto \\
res!: RES
\where
nombProd? \in \dom (clientes \: mail?).productos \\
nombProd? \in \dom stock \\
cant? > 0 \\
cant? \leq ((clientes \: mail?).productos \: nombProd?) + (stock \: nombProd?) \\
stock' = \\ 
\t1 stock \\
\t1 \oplus \{ nombProd? \mapsto \\
\t2 (stock \: nombProd?) - (cant? - ((clientes \: mail?).productos \: nombProd?)) \} \\
res! = ok
\end{schema}

\begin{schema}{SCEditarProductoEliminar}
CarroASistemaDeCompra \\
CEliminarProducto \\
cant?: CANTIDAD \\
res!: RES
\where
nombProd? \in \dom (clientes \: mail?).productos \\
nombProd? \in \dom stock \\
cant? = 0 \\
stock' = \\ 
\t1 stock \\
\t1 \oplus \{ nombProd? \mapsto \\
\t2 (stock \: nombProd?) + ((clientes \: mail?).productos \: nombProd?) \} \\
res! = ok
\end{schema}

\begin{schema}{SCEditarAumentarProductoError}
\Xi SistemaDeCompra \\
CarroASistemaDeCompra \\
nombProd?: NOMBREPRODUCTO \\
cant?: CANTIDAD \\
res!: RES
\where
nombProd? \in \dom (clientes \: mail?).productos \\
nombProd? \in \dom stock \\
cant? > 0 \\
cant? > ((clientes \: mail?).productos \: nombProd?) + (stock \: nombProd?) \\
res! = errorNoHayStock
\end{schema}

\begin{schema}{SCEditarProductoErrorProductoNoHayStock}
\Xi SistemaDeCompra \\
CarroASistemaDeCompra \\
nombProd?: NOMBREPRODUCTO \\
cant?: CANTIDAD \\
res!: RES
\where
nombProd? \in \dom (clientes \: mail?).productos \\
nombProd? \notin \dom stock \\
res! = errorNoHayStock
\end{schema}

\begin{zed}
SCEditarAumentarProducto == SCEditarAumentarProductoOk \\
\t1 \lor SCEditarAumentarProductoError \\
SCEditarProductoError == ErrorClienteNoIniciado \\
\t1 \lor ErrorProductoNoExisteEnCarro \\
\t1 \lor SCEditarProductoErrorProductoNoHayStock \\
SCEditarProducto == SCEditarAumentarProducto \\
\t1 \lor SCEditarDisminuirProductoOk \\
\t1 \lor SCEditarProductoEliminar \\
\t1 \lor SCEditarProductoError 
\end{zed}

\begin{schema}{CarroVacio}
Carro
\where
productos = \emptyset
\end{schema}

\begin{schema}{SCIniciarCompraOk}
\Delta SistemaDeCompra \\
CarroVacio \\
mail?: MAIL \\
res!: RES
\where
mail? \notin \dom clientes \\
clientes' = clientes \cup \{ mail? \mapsto \theta Carro \} \\
stock' = stock \\
res! = ok
\end{schema}

\begin{schema}{SCIniciarCompraError}
\Xi SistemaDeCompra \\
CarroVacio \\
mail?: MAIL \\
res!: RES
\where
mail? \in \dom clientes \\
res! = errorClienteIniciado
\end{schema}

\begin{zed}
SCIniciarCompra == SCIniciarCompraOk \lor SCIniciarCompraError
\end{zed}

\begin{schema}{SCFinalizarCompraOk}
\Delta SistemaDeCompra \\
mail?: MAIL \\
res!: RES
\where
mail? \in \dom clientes \\
clientes' = \{ mail? \} \ndres clientes \\
stock' = stock \\
res! = ok
\end{schema}

\begin{schema}{SCFinalizarCompraError}
\Xi SistemaDeCompra \\
mail?: MAIL \\
res!: RES
\where
mail? \notin \dom clientes \\
res! = errorClienteNoIniciado
\end{schema}

\begin{zed}
SCFinalizarCompra == SCFinalizarCompraOk \lor SCFinalizarCompraError
\end{zed}

\end{document}