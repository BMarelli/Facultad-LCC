\documentclass{article}
\usepackage{czt}

\begin{document}
\begin{zsection}
  \SECTION ejercicio2 \parents standard\_toolkit
\end{zsection}

\section{Apartado 2}
\begin{zed}
  [MENSAJE] \\
  ESTADO ::= enviando | recibiendo \\
  ERROR ::= bufferLleno | errorPariedad
\end{zed}

\begin{axdef}
MAXIMOBUF: \nat
\where
MAXIMOBUF = 10
\end{axdef}

\begin{axdef}
PARIEDAD: \power MENSAJE
\where
PARIEDAD \neq \emptyset
\end{axdef}

\begin{schema}{Dispacher}
estado : ESTADO \\
buf: \seq MENSAJE \\
pariedad: \power MENSAJE
\end{schema}

\begin{schema}{DispacherInit}
Dispacher
\where
pariedad = \emptyset \\
estado = recibiendo \\
buf = \langle  \rangle
\end{schema}

\begin{schema}{Canal1Ok}
\Delta Dispacher \\
m?: MENSAJE \\
\where
m? \in pariedad \\
estado = recibiendo \\
\# buf \leq MAXIMOBUF \\
buf' = buf \cat \langle m? \rangle \\
estado' = estado \\
pariedad' = pariedad
\end{schema}

\begin{schema}{Canal1ErrorLleno}
\Xi Dispacher \\
m?: MENSAJE \\
rep!: ERROR
\where
m? \in pariedad \\
estado = recibiendo \\
\# buf = MAXIMOBUF \\
rep! = bufferLleno
\end{schema}

\begin{schema}{BufferLleno}
\Delta Dispacher \\
\where
estado = recibiendo \\
\# buf = MAXIMOBUF \\
buf' = buf \\
pariedad' = pariedad \\
estado' = enviando
\end{schema}

\begin{schema}{Canal1ErrorPariedad}
\Xi Dispacher \\
m?: MENSAJE \\
rep!: ERROR
\where
m? \notin pariedad \\
rep! = errorPariedad
\end{schema}

\begin{zed}
EnviarMensajeCanal1 == Canal1Ok \lor Canal1ErrorLleno \lor Canal1ErrorPariedad
\end{zed}

% Porque no me deja hacer rep!: MENSAJE
\begin{schema}{VaciandoBuffer}
\Delta Dispacher \\
rep!: MENSAJE
\where
estado = enviando \\
\# buf \geq 1 \\
buf' = tail \: buf \\
rep! = head \: buf
\end{schema}

\begin{schema}{BufferVacio}
\Delta Dispacher \\
\where
estado = enviando \\
\# buf = 0 \\
estado' = recibiendo \\
pariedad' = pariedad \\
buf' = buf
\end{schema}

\begin{zed}
VaciarBuf == VaciandoBuffer \lor BufferVacio
\end{zed}

\end{document}